\documentclass[10pt,letterpaper]{article}

\usepackage[margin=1in]{geometry}
\usepackage{newtxtext,newtxmath}
\usepackage[final,protrusion=true,expansion=true]{microtype}
\usepackage{graphicx,booktabs,caption,enumitem,titlesec,titling}
\setlength{\parskip}{0pt}
\setlength{\parindent}{1.2em}
\titleformat{\section}{\bfseries}{\thesection}{0.6em}{}   
\titlespacing*{\section}{0pt}{6pt}{3pt}
\setlist{nosep,leftmargin=*,itemsep=0pt,topsep=2pt}
\captionsetup{font=small,labelfont=bf,aboveskip=4pt,belowskip=0pt}
\setlength{\tabcolsep}{5pt}
\raggedbottom
\pagestyle{empty}

\begin{document}

\begin{center}
\textbf{James Rogan, 261157151. Fall 2025}\\
\textbf{ECSE 324 Lab 3 Report}\\
\textbf{Comprative analysis of polling and interupts/timers in ARMv7 Assembly}
\end{center}

\section{Introduction}
An interactive rotating hex display was implemented in \textit{ARMv7 assembly} using both polling and interrupts/timers and profiled on \textit{CPUlator}. 

\section{Methodology}

\subsection{Time Fraction Measurement}

To measure what fraction of time is spent on different activities, we use the total \textit{instructions executed} as a proxy for elapsed time. While not perfectly accurate (since different instructions take different cycles on real hardware), the emulator uniformly reports instruction counts, making this a practical metric.

\subsubsection{Measurement Approach}

The methodology uses CPUlator's debugger breakpoints to establish instruction count checkpoints at key program locations:

\begin{enumerate}
    \item \textbf{Identify key code regions}:
    \begin{itemize}
        \item Interrupt service routine (ISR) entry point
        \item ISR exit point
        \item Slider switch polling loop entry
        \item Slider switch polling loop exit
        \item Main/IDLE code regions
    \end{itemize}

    \item \textbf{Place strategic breakpoints}: Set breakpoints at entry and exit points of each region of interest.

    \item \textbf{Record instruction counts}: Note the cumulative instruction count displayed by CPUlator at each breakpoint.

    \item \textbf{Calculate region instruction counts}:
    \begin{itemize}
        \item Instructions in ISR = (ISR exit count) -- (ISR entry count)
        \item Instructions in polling = (polling exit count) -- (polling entry count)
        \item Instructions in user/IDLE code = Total -- Instructions in ISR -- Instructions in polling
    \end{itemize}

    \item \textbf{Calculate time fractions}:
    \begin{align}
        \text{Fraction}_{\text{ISR}} &= \frac{\text{Instructions}_{\text{ISR}}}{\text{Total Instructions}} \\
        \text{Fraction}_{\text{polling}} &= \frac{\text{Instructions}_{\text{polling}}}{\text{Total Instructions}} \\
        \text{Fraction}_{\text{IDLE/user}} &= \frac{\text{Instructions}_{\text{IDLE/user}}}{\text{Total Instructions}}
    \end{align}
\end{enumerate}

\subsubsection{Practical Implementation}

In CPUlator:
\begin{enumerate}
    \item Run the program to a representative steady state (allow several cycles of activity)
    \item Set a breakpoint at the start of the ISR
    \item Record the starting instruction count before sampling begins
    \item Sample for a fixed duration (e.g., 1,000,000 or 10,000,000 total instructions)
    \item Place breakpoints within interrupt handlers and polling loops
    \item Let the program run to completion while collecting checkpoint data
    \item Calculate the instruction deltas for each region
\end{enumerate}

This approach leverages the emulator's cycle-accurate instruction counting to provide reproducible measurements without requiring external profiling tools.

\section{Results} % TODO, maybe add multiplr tables of results
\begin{table}[h]
\centering
\caption{Example table for comparisons of the metrics that CPUlator provides}
\begin{tabular}{lrrrr}
\hline
\textbf{Metric} & \textbf{...} & \textbf{...} & \textbf{Improvement} \\
\hline
Executed instructions &    &  & 3.29$\times$ \\
Data loads            &   &  & 3.97$\times$ \\
Data stores           &   &      & 3.93$\times$ \\
Code size (bytes)     &   &      & 2.38$\times$ \\
Simulated MIPS     &   &      & 3.29$\times$ \\
\hline
\end{tabular}
\end{table}

\section{Comparison of clock cycles (or other title)} % TODO compare clock cycles

\section{Discussion} % TODO complete according to the 

\section{Conclusion} % TODO complete according to appropriate conclusions from the data

\end{document}
